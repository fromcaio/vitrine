%% ===================================================================================
%% Apêndices e Anexos (appendix_a.tex)
%% Versão: 2.2 (Corrigida)
%%
%% Este arquivo contém os elementos pós-textuais do seu trabalho. Ele separa
%% claramente Apêndices (trabalho do próprio autor) de Anexos (material de terceiros).
%%
%% Para usar este arquivo:
%% 1. Adicione seu conteúdo sob as seções apropriadas abaixo.
%% 2. No arquivo main.tex, descomente os blocos \appendix ou \anexos e use
%%    %% ===================================================================================
%% Apêndices e Anexos (appendix_a.tex)
%% Versão: 2.2 (Corrigida)
%%
%% Este arquivo contém os elementos pós-textuais do seu trabalho. Ele separa
%% claramente Apêndices (trabalho do próprio autor) de Anexos (material de terceiros).
%%
%% Para usar este arquivo:
%% 1. Adicione seu conteúdo sob as seções apropriadas abaixo.
%% 2. No arquivo main.tex, descomente os blocos \appendix ou \anexos e use
%%    %% ===================================================================================
%% Apêndices e Anexos (appendix_a.tex)
%% Versão: 2.2 (Corrigida)
%%
%% Este arquivo contém os elementos pós-textuais do seu trabalho. Ele separa
%% claramente Apêndices (trabalho do próprio autor) de Anexos (material de terceiros).
%%
%% Para usar este arquivo:
%% 1. Adicione seu conteúdo sob as seções apropriadas abaixo.
%% 2. No arquivo main.tex, descomente os blocos \appendix ou \anexos e use
%%    %% ===================================================================================
%% Apêndices e Anexos (appendix_a.tex)
%% Versão: 2.2 (Corrigida)
%%
%% Este arquivo contém os elementos pós-textuais do seu trabalho. Ele separa
%% claramente Apêndices (trabalho do próprio autor) de Anexos (material de terceiros).
%%
%% Para usar este arquivo:
%% 1. Adicione seu conteúdo sob as seções apropriadas abaixo.
%% 2. No arquivo main.tex, descomente os blocos \appendix ou \anexos e use
%%    \include{posttextual/appendix_a} para adicionar este conteúdo ao seu documento.
%% ===================================================================================

% -----------------------------------------------------------------------------------
% -- APÊNDICES ----------------------------------------------------------------------
% Um Apêndice contém material suplementar CRIADO PELO AUTOR.
% Exemplos: código-fonte, dados de pesquisa, provas matemáticas detalhadas.
% -----------------------------------------------------------------------------------
% \appendix % Este comando inicia a seção de apêndices, numerando os capítulos com letras (A, B, C...).

% [A FAZER: Adicione seus apêndices abaixo. Cada apêndice deve ser um novo \chapter.]
% Se você não tiver apêndices, pode deletar esta seção.

\chapter{Código Fonte do Módulo de Autenticação}
\label{ap:codigo_fonte}

Este apêndice contém o código-fonte completo para o módulo de autenticação discutido no Capítulo 4. O código é fornecido para referência e completude.

Você pode adicionar blocos de código com realce de sintaxe usando o ambiente `minted`.
\begin{minted}[linenos, frame=lines]{python}
# auth_module.py

def validar_usuario(username, password):
    """
    Valida as credenciais do usuário contra o banco de dados.
    Esta é uma função de exemplo.
    """
    if not username or not password:
        return False
    # Em uma aplicação real, você consultaria um banco de dados aqui.
    print(f"Validando usuário: {username}")
    return True
\end{minted}


% -----------------------------------------------------------------------------------
% -- ANEXOS -------------------------------------------------------------------------
% Um Anexo contém material de apoio DE OUTRAS FONTES.
% Exemplos: mapas, leis, manuais técnicos de um fabricante.
% -----------------------------------------------------------------------------------
% \anexos % Este comando inicia a seção de anexos, continuando a numeração com letras.

% [A FAZER: Adicione seus anexos abaixo. Cada anexo deve ser um novo \chapter.]
% Se você não tiver anexos, pode deletar esta seção.

\chapter{Manual Técnico do Sensor Y}
\label{anx:manual}

Este anexo fornece o manual técnico oficial do fabricante para o sensor de temperatura utilizado no experimento (Capítulo 3). O documento especifica sua faixa de operação e precisão.

Você pode incluir um documento externo (como um PDF ou imagem) como uma figura.
\begin{figure}[H] % A opção [H] do pacote 'float' posiciona a figura exatamente aqui.
    \centering
    % \includegraphics[width=0.8\textwidth]{caminho/para/imagem_manual.png}
    \fbox{Espaço reservado para uma imagem ou documento escaneado.}
    \caption{Especificações técnicas do Sensor Y.} para adicionar este conteúdo ao seu documento.
%% ===================================================================================

% -----------------------------------------------------------------------------------
% -- APÊNDICES ----------------------------------------------------------------------
% Um Apêndice contém material suplementar CRIADO PELO AUTOR.
% Exemplos: código-fonte, dados de pesquisa, provas matemáticas detalhadas.
% -----------------------------------------------------------------------------------
% \appendix % Este comando inicia a seção de apêndices, numerando os capítulos com letras (A, B, C...).

% [A FAZER: Adicione seus apêndices abaixo. Cada apêndice deve ser um novo \chapter.]
% Se você não tiver apêndices, pode deletar esta seção.

\chapter{Código Fonte do Módulo de Autenticação}
\label{ap:codigo_fonte}

Este apêndice contém o código-fonte completo para o módulo de autenticação discutido no Capítulo 4. O código é fornecido para referência e completude.

Você pode adicionar blocos de código com realce de sintaxe usando o ambiente `minted`.
\begin{minted}[linenos, frame=lines]{python}
# auth_module.py

def validar_usuario(username, password):
    """
    Valida as credenciais do usuário contra o banco de dados.
    Esta é uma função de exemplo.
    """
    if not username or not password:
        return False
    # Em uma aplicação real, você consultaria um banco de dados aqui.
    print(f"Validando usuário: {username}")
    return True
\end{minted}


% -----------------------------------------------------------------------------------
% -- ANEXOS -------------------------------------------------------------------------
% Um Anexo contém material de apoio DE OUTRAS FONTES.
% Exemplos: mapas, leis, manuais técnicos de um fabricante.
% -----------------------------------------------------------------------------------
% \anexos % Este comando inicia a seção de anexos, continuando a numeração com letras.

% [A FAZER: Adicione seus anexos abaixo. Cada anexo deve ser um novo \chapter.]
% Se você não tiver anexos, pode deletar esta seção.

\chapter{Manual Técnico do Sensor Y}
\label{anx:manual}

Este anexo fornece o manual técnico oficial do fabricante para o sensor de temperatura utilizado no experimento (Capítulo 3). O documento especifica sua faixa de operação e precisão.

Você pode incluir um documento externo (como um PDF ou imagem) como uma figura.
\begin{figure}[H] % A opção [H] do pacote 'float' posiciona a figura exatamente aqui.
    \centering
    % \includegraphics[width=0.8\textwidth]{caminho/para/imagem_manual.png}
    \fbox{Espaço reservado para uma imagem ou documento escaneado.}
    \caption{Especificações técnicas do Sensor Y.} para adicionar este conteúdo ao seu documento.
%% ===================================================================================

% -----------------------------------------------------------------------------------
% -- APÊNDICES ----------------------------------------------------------------------
% Um Apêndice contém material suplementar CRIADO PELO AUTOR.
% Exemplos: código-fonte, dados de pesquisa, provas matemáticas detalhadas.
% -----------------------------------------------------------------------------------
% \appendix % Este comando inicia a seção de apêndices, numerando os capítulos com letras (A, B, C...).

% [A FAZER: Adicione seus apêndices abaixo. Cada apêndice deve ser um novo \chapter.]
% Se você não tiver apêndices, pode deletar esta seção.

\chapter{Código Fonte do Módulo de Autenticação}
\label{ap:codigo_fonte}

Este apêndice contém o código-fonte completo para o módulo de autenticação discutido no Capítulo 4. O código é fornecido para referência e completude.

Você pode adicionar blocos de código com realce de sintaxe usando o ambiente `minted`.
\begin{minted}[linenos, frame=lines]{python}
# auth_module.py

def validar_usuario(username, password):
    """
    Valida as credenciais do usuário contra o banco de dados.
    Esta é uma função de exemplo.
    """
    if not username or not password:
        return False
    # Em uma aplicação real, você consultaria um banco de dados aqui.
    print(f"Validando usuário: {username}")
    return True
\end{minted}


% -----------------------------------------------------------------------------------
% -- ANEXOS -------------------------------------------------------------------------
% Um Anexo contém material de apoio DE OUTRAS FONTES.
% Exemplos: mapas, leis, manuais técnicos de um fabricante.
% -----------------------------------------------------------------------------------
% \anexos % Este comando inicia a seção de anexos, continuando a numeração com letras.

% [A FAZER: Adicione seus anexos abaixo. Cada anexo deve ser um novo \chapter.]
% Se você não tiver anexos, pode deletar esta seção.

\chapter{Manual Técnico do Sensor Y}
\label{anx:manual}

Este anexo fornece o manual técnico oficial do fabricante para o sensor de temperatura utilizado no experimento (Capítulo 3). O documento especifica sua faixa de operação e precisão.

Você pode incluir um documento externo (como um PDF ou imagem) como uma figura.
\begin{figure}[H] % A opção [H] do pacote 'float' posiciona a figura exatamente aqui.
    \centering
    % \includegraphics[width=0.8\textwidth]{caminho/para/imagem_manual.png}
    \fbox{Espaço reservado para uma imagem ou documento escaneado.}
    \caption{Especificações técnicas do Sensor Y.} para adicionar este conteúdo ao seu documento.
%% ===================================================================================

% -----------------------------------------------------------------------------------
% -- APÊNDICES ----------------------------------------------------------------------
% Um Apêndice contém material suplementar CRIADO PELO AUTOR.
% Exemplos: código-fonte, dados de pesquisa, provas matemáticas detalhadas.
% -----------------------------------------------------------------------------------
% \appendix % Este comando inicia a seção de apêndices, numerando os capítulos com letras (A, B, C...).

% [A FAZER: Adicione seus apêndices abaixo. Cada apêndice deve ser um novo \chapter.]
% Se você não tiver apêndices, pode deletar esta seção.

\chapter{Código Fonte do Módulo de Autenticação}
\label{ap:codigo_fonte}

Este apêndice contém o código-fonte completo para o módulo de autenticação discutido no Capítulo 4. O código é fornecido para referência e completude.

Você pode adicionar blocos de código com realce de sintaxe usando o ambiente `minted`.
\begin{minted}[linenos, frame=lines]{python}
# auth_module.py

def validar_usuario(username, password):
    """
    Valida as credenciais do usuário contra o banco de dados.
    Esta é uma função de exemplo.
    """
    if not username or not password:
        return False
    # Em uma aplicação real, você consultaria um banco de dados aqui.
    print(f"Validando usuário: {username}")
    return True
\end{minted}


% -----------------------------------------------------------------------------------
% -- ANEXOS -------------------------------------------------------------------------
% Um Anexo contém material de apoio DE OUTRAS FONTES.
% Exemplos: mapas, leis, manuais técnicos de um fabricante.
% -----------------------------------------------------------------------------------
% \anexos % Este comando inicia a seção de anexos, continuando a numeração com letras.

% [A FAZER: Adicione seus anexos abaixo. Cada anexo deve ser um novo \chapter.]
% Se você não tiver anexos, pode deletar esta seção.

\chapter{Manual Técnico do Sensor Y}
\label{anx:manual}

Este anexo fornece o manual técnico oficial do fabricante para o sensor de temperatura utilizado no experimento (Capítulo 3). O documento especifica sua faixa de operação e precisão.

Você pode incluir um documento externo (como um PDF ou imagem) como uma figura.
\begin{figure}[H] % A opção [H] do pacote 'float' posiciona a figura exatamente aqui.
    \centering
    % \includegraphics[width=0.8\textwidth]{caminho/para/imagem_manual.png}
    \fbox{Espaço reservado para uma imagem ou documento escaneado.}
    \caption{Especificações técnicas do Sensor Y.}