%% ===================================================================================
%% Template para Trabalho de Conclusão de Curso (TCC) utilizando abntex2
%% Versão: 2.2 (Corrigida)
%% Baseado no modelo original abntex2 do grupo abnTeX2 e derivados.
%% Este template foi projetado para ser um ponto de partida limpo, moderno e 
%% fácil de usar para trabalhos de graduação, seguindo as normas da ABNT.
%% ===================================================================================

% A classe 'undergrad_thesis' é baseada na 'abntex2', configurada para as normas da ABNT.
\documentclass[
	% -- Opções da classe Memoir --
	12pt,      % Tamanho da fonte
	openright, % Capítulos começam em páginas ímpares (insere página em branco se necessário)
	oneside,   % Para impressão em um lado. Use 'twoside' para frente e verso.
	a4paper,   % Tamanho do papel
	% -- Opções da classe abntex2 --
	% Descomente para colocar os títulos de capítulos, seções, etc., em maiúsculas
	% chapter=TITLE,
	% section=TITLE,
	% subsection=TITLE,
	% -- Opções do pacote Babel --
	english,   % Idioma adicional para hifenização
	brazil     % O último idioma é o principal do documento
]{undergrad_thesis}

%% ===================================================================================
%% IMPORTAÇÃO DE PACOTES
%% ===================================================================================

% -----------------------------------------------------------------------------------
% -- Pacotes Fundamentais
% -----------------------------------------------------------------------------------
\usepackage[T1]{fontenc}      % Usa fontes T1 de 8-bits
\usepackage[utf8]{inputenc}   % Codificação UTF-8 para acentos e caracteres especiais
\usepackage{lmodern}          % Usa a família de fontes Latin Modern para melhor renderização do PDF
\usepackage{indentfirst}      % Indenta o primeiro parágrafo de cada seção
\usepackage{cmap}             % Torna o PDF pesquisável e copiável

% -----------------------------------------------------------------------------------
% -- Gráficos e Cores
% -----------------------------------------------------------------------------------
\usepackage{graphicx}         % Para inclusão de imagens
\usepackage{subcaption}       % Para criar subfiguras dentro de um ambiente figure

% -----------------------------------------------------------------------------------
% -- Tabelas e Listagens de Código
% -----------------------------------------------------------------------------------
\usepackage{booktabs}         % Para tabelas com qualidade profissional
\usepackage{listings}         % Para listagens de código básicas
\usepackage{minted}           % Para listagens de código avançadas com realce de sintaxe (requer Pygments do Python)
\usepackage[table]{xcolor}    % Para adicionar cor a linhas/células de tabelas
\usepackage{multirow}         % Para mesclar linhas em tabelas

% -----------------------------------------------------------------------------------
% -- Matemática
% -----------------------------------------------------------------------------------
\usepackage{amsmath, amsfonts, amssymb} % Para composição matemática avançada

% -----------------------------------------------------------------------------------
% -- Formatação de Código
% -----------------------------------------------------------------------------------
\usepackage{inconsolata} % Fonte monoespaçada de boa legibilidade
\lstset{
    numbers=left,                   % Numeração de linhas à esquerda
    numberstyle=\tiny,              % Estilo dos números de linha
    numbersep=8pt,                  % Espaço entre os números e o código
    frame=single,                   % Borda ao redor do código
    framesep=6pt,                   % Espaço entre a borda e o código
    xleftmargin=2em,                % Margem esquerda para acomodar os números
    xrightmargin=1em,               % Margem direita
    breaklines=true,                % Quebra de linhas longas
    postbreak=\mbox{$\hookrightarrow$\space}, % Símbolo para indicar quebra de linha
    basicstyle=\linespread{1.05}\ttfamily\footnotesize, % Estilo base do texto
    keywordstyle=,                  % Estilo das palavras-chave (pode ser customizado)
    commentstyle=,                  % Estilo dos comentários (pode ser customizado)
    stringstyle=,                   % Estilo das strings (pode ser customizado)
    showstringspaces=false,         % Não exibe espaços em strings de forma especial
    tabsize=2,                      % Tamanho da tabulação
    captionpos=b,                   % Posição da legenda (bottom)
    rulecolor=,                     % Cor da borda (pode ser customizado)
}

% -----------------------------------------------------------------------------------
% -- Citações e Bibliografia
% -----------------------------------------------------------------------------------
\usepackage[num]{abntex2cite}  % Estilo de citação ABNT (numérico)
% Para o estilo autor-data, use: \usepackage[alf]{abntex2cite}

%% ===================================================================================
%% COMANDOS E CONFIGURAÇÕES PERSONALIZADAS
%% ===================================================================================

% -----------------------------------------------------------------------------------
% -- Informações do Documento (para capa, folha de rosto e metadados do PDF)
% -----------------------------------------------------------------------------------
% [A FAZER: Preencha com as informações do seu trabalho]
\titulo{Título do Trabalho de Conclusão de Curso}
\autor{Seu Nome Completo}
\local{Cidade, Estado}
\data{\the\year}
\orientador{Prof. Dr. Nome do Orientador}
% \coorientador{Prof. Dr. Nome do Coorientador} % Descomente se houver coorientador
\instituicao{%
  Nome da Universidade
  \par
  Curso de Graduação em Nome do Curso}
\tipotrabalho{Trabalho de Conclusão de Curso} % Ou Monografia, Relatório, etc.
\preambulo{Trabalho de Conclusão de Curso apresentado como requisito parcial para a obtenção do grau de Bacharel em Nome do Curso pela Universidade Nome da Universidade.}
\logo{imgs/university-logo.jpg} % [A FAZER: Adicione o logo da sua universidade na pasta 'imgs']

% -----------------------------------------------------------------------------------
% -- Configuração de Hiperlinks e Metadados do PDF
% -----------------------------------------------------------------------------------
\definecolor{hyperlinkblue}{RGB}{41, 5, 195} % Cor personalizada para os links

\makeatletter
\hypersetup{
    pdftitle={\@title},
    pdfauthor={\@author},
    pdfsubject={\imprimirpreambulo},
    pdfcreator={LaTeX com abnTeX2},
    pdfkeywords={palavra-chave1}{palavra-chave2}{palavra-chave3},
    colorlinks=true,
    linkcolor=hyperlinkblue,
    citecolor=hyperlinkblue,
    filecolor=magenta,
    urlcolor=hyperlinkblue,
    bookmarksdepth=4,
    bookmarkstype=toc
}
\makeatother

% -----------------------------------------------------------------------------------
% -- Espaçamento e Layout
% -----------------------------------------------------------------------------------
\setlength{\parindent}{1.5cm} % Indentação do parágrafo
\setlength{\parskip}{0.2cm}   % Espaçamento vertical entre parágrafos

% -----------------------------------------------------------------------------------
% -- Índice Remissivo
% -----------------------------------------------------------------------------------
\makeindex % Gera o arquivo do índice remissivo

%% ===================================================================================
%% INÍCIO DO DOCUMENTO
%% ===================================================================================
\begin{document}

% Usa os padrões de hifenização e espaçamento do português do Brasil
\frenchspacing

% ----------------------------------------------------------
% -- ELEMENTOS PRÉ-TEXTUAIS
% ----------------------------------------------------------
\pretextual

% --- Capa e Folha de Rosto ---
\imprimircapalogo % Use \imprimircapa se não quiser o logo
\imprimirfolhaderosto

% --- Outros elementos pré-textuais (folha de aprovação, dedicatória, etc.) ---
% [A FAZER: Edite o arquivo 'pretextual_elements.tex' com suas informações]
%% ===================================================================================
%% Elementos Pré-Textuais (pretextual_elements.tex)
%% Versão: 2.2 (Corrigida)
%%
%% Este arquivo contém:
%% - Folha de Aprovação
%% - Dedicatória
%% - Agradecimentos
%% - Epígrafe
%% - Resumo em Português
%% - Resumo em Língua Estrangeira (ex: Inglês)
%%
%% Preencha, descomente ou remova as seções conforme sua necessidade.
%% ===================================================================================


% -----------------------------------------------------------------------------------
% -- FOLHA DE APROVAÇÃO
% -----------------------------------------------------------------------------------
% NOTA: Este é um elemento obrigatório (ABNT NBR 14724:2011).
% Você pode usar esta versão em texto para rascunhos. Após a defesa, é uma
% prática comum escanear a página assinada e substituir todo o conteúdo do
% ambiente 'folhadeaprovacao' por: \includegraphics[width=\textwidth]{caminho/para/pagina_assinada.pdf}
% -----------------------------------------------------------------------------------
\begin{folhadeaprovacao}
  \begin{center}
    {\ABNTEXchapterfont\large\imprimirautor}

    \vspace*{\fill}
    {\ABNTEXchapterfont\bfseries\Large\imprimirtitulo}
    \vspace*{\fill}

    \hspace{.45\textwidth}
    \begin{minipage}{.5\textwidth}
      \SingleSpacing
      \imprimirpreambulo
    \end{minipage}%
    \vspace*{\fill}
  \end{center}

  % [A FAZER: Ajuste a data de aprovação. \today usa a data da compilação.]
  Trabalho aprovado. \imprimirlocal, \today:
  \vspace*{2cm}

  % [A FAZER: Preencha com os nomes dos membros da sua banca examinadora.]
  \assinatura{\textbf{\imprimirorientador} \\ Orientador(a)}
  \assinatura{\textbf{Nome Completo Membro da Banca 1} \\ Instituição}
  \assinatura{\textbf{Nome Completo Membro da Banca 2} \\ Instituição}
  % \assinatura{\textbf{Nome Completo do Membro da Banca 3} \\ Instituição} % Descomente se necessário

\end{folhadeaprovacao}


% -----------------------------------------------------------------------------------
% -- DEDICATÓRIA (Opcional)
% -----------------------------------------------------------------------------------
% [A FAZER: Escreva sua dedicatória aqui ou comente/remova todo o ambiente.]
\begin{dedicatoria}
	\vspace*{\fill} % Centraliza o texto verticalmente
	\flushright    % Alinha o texto à direita
	\textit{DEDICATÓRIA: Agradeço aos ..., pelo apoio e incentivo incondicional...}
\end{dedicatoria}


% -----------------------------------------------------------------------------------
% -- AGRADECIMENTOS (Opcional)
% -----------------------------------------------------------------------------------
% [A FAZER: Escreva seus agradecimentos aqui ou comente/remova todo o ambiente.]
\begin{agradecimentos}
Agradeço a todos que contribuíram para a realização deste trabalho. Em especial, ao meu orientador, Prof. Dr. [Nome do Orientador], pelo suporte e orientação ao longo de todo o processo.

\end{agradecimentos}


% -----------------------------------------------------------------------------------
% -- EPÍGRAFE (Opcional)
% -----------------------------------------------------------------------------------
% [A FAZER: Adicione uma epígrafe aqui ou comente/remova todo o ambiente.]
\begin{epigrafe}
	\vspace*{\fill}
	\begin{flushright}
		\textit{``A educação é a arma mais poderosa que você pode usar para mudar o mundo."} \\
		\textendash{} Nelson Mandela
	\end{flushright}
\end{epigrafe}


% -----------------------------------------------------------------------------------
% -- RESUMO EM PORTUGUÊS (Obrigatório)
% -----------------------------------------------------------------------------------
\begin{resumo}
	% [A FAZER: Escreva o resumo do seu trabalho em português aqui.]
	O resumo deve apresentar de forma concisa o tema do trabalho, o problema de pesquisa, os objetivos, a metodologia utilizada e os principais resultados ou conclusões alcançadas. Ele é uma parte crucial do trabalho, pois fornece uma visão geral que permite ao leitor compreender rapidamente o escopo e a contribuição da sua pesquisa. O resumo deve ser escrito em um único parágrafo, contendo entre 150 e 500 palavras, conforme a ABNT.

	\vspace{\onelineskip} % Adiciona uma linha em branco antes das palavras-chave
	\noindent % Impede a indentação da linha das palavras-chave
	\textbf{Palavras-chave}: Engenharia de Software; Metodologia Ágil; Qualidade de Software.
\end{resumo}


% -----------------------------------------------------------------------------------
% -- RESUMO EM LÍNGUA ESTRANGEIRA (Obrigatório)
% -----------------------------------------------------------------------------------
\begin{resumo}[Abstract] % O título do resumo no idioma desejado
	\begin{otherlanguage*}{english} % Define o idioma para este ambiente

		% [A FAZER: Escreva o resumo do seu trabalho em inglês aqui.]
		The abstract should concisely present the theme of the work, the research problem, objectives, the methodology used, and the main results or conclusions reached. It is a crucial part of the work, as it provides an overview that allows the reader to quickly understand the scope and contribution of your research. The abstract should be written in a single paragraph, containing between 150 and 500 words, according to ABNT standards.

		\vspace{\onelineskip}
		\noindent
		\textbf{Keywords}: Software Engineering; Agile Methodology; Software Quality.

	\end{otherlanguage*}
\end{resumo}

% --- Lista de Figuras ---
\pdfbookmark[0]{\listfigurename}{lof}
\listoffigures*
\cleardoublepage

% --- Lista de Tabelas ---
% [A FAZER: Descomente as linhas abaixo se seu trabalho tiver tabelas]
% \pdfbookmark[0]{\listtablename}{lot}
% \listoftables*
% \cleardoublepage

% --- Lista de Siglas e Abreviaturas ---
% [A FAZER: Adicione suas siglas em 'acronyms.tex' e descomente a linha abaixo]
%% ===================================================================================
%% Siglas e Abreviaturas (acronyms.tex)
%% Versão: 2.2 (Corrigida)
%%
%% Este arquivo contém a lista de siglas utilizadas no trabalho.
%%
%% Para usar este arquivo:
%% 1. Adicione suas siglas abaixo seguindo o formato do exemplo.
%% 2. No arquivo main.tex, descomente a linha que diz: %% ===================================================================================
%% Siglas e Abreviaturas (acronyms.tex)
%% Versão: 2.2 (Corrigida)
%%
%% Este arquivo contém a lista de siglas utilizadas no trabalho.
%%
%% Para usar este arquivo:
%% 1. Adicione suas siglas abaixo seguindo o formato do exemplo.
%% 2. No arquivo main.tex, descomente a linha que diz: %% ===================================================================================
%% Siglas e Abreviaturas (acronyms.tex)
%% Versão: 2.2 (Corrigida)
%%
%% Este arquivo contém a lista de siglas utilizadas no trabalho.
%%
%% Para usar este arquivo:
%% 1. Adicione suas siglas abaixo seguindo o formato do exemplo.
%% 2. No arquivo main.tex, descomente a linha que diz: \include{acronyms}
%% ===================================================================================

% O ambiente 'siglas' cria a lista de siglas e abreviaturas.
\begin{siglas}
	% [A FAZER: Adicione as siglas e abreviaturas do seu projeto aqui.]
	% O formato é: \item[SIGLA] Nome Completo / Descrição
	
	\item[ABNT] Associação Brasileira de Normas Técnicas
	\item[API] Interface de Programação de Aplicações
	\item[BD] Banco de Dados
	\item[HTML] Linguagem de Marcação de Hipertexto
	\item[IDE] Ambiente de Desenvolvimento Integrado
	\item[JSON] Notação de Objetos JavaScript
	\item[SQL] Linguagem de Consulta Estrurada
	\item[UCP] Unidade Central de Processamento
	
\end{siglas}

%% ===================================================================================

% O ambiente 'siglas' cria a lista de siglas e abreviaturas.
\begin{siglas}
	% [A FAZER: Adicione as siglas e abreviaturas do seu projeto aqui.]
	% O formato é: \item[SIGLA] Nome Completo / Descrição
	
	\item[ABNT] Associação Brasileira de Normas Técnicas
	\item[API] Interface de Programação de Aplicações
	\item[BD] Banco de Dados
	\item[HTML] Linguagem de Marcação de Hipertexto
	\item[IDE] Ambiente de Desenvolvimento Integrado
	\item[JSON] Notação de Objetos JavaScript
	\item[SQL] Linguagem de Consulta Estrurada
	\item[UCP] Unidade Central de Processamento
	
\end{siglas}

%% ===================================================================================

% O ambiente 'siglas' cria a lista de siglas e abreviaturas.
\begin{siglas}
	% [A FAZER: Adicione as siglas e abreviaturas do seu projeto aqui.]
	% O formato é: \item[SIGLA] Nome Completo / Descrição
	
	\item[ABNT] Associação Brasileira de Normas Técnicas
	\item[API] Interface de Programação de Aplicações
	\item[BD] Banco de Dados
	\item[HTML] Linguagem de Marcação de Hipertexto
	\item[IDE] Ambiente de Desenvolvimento Integrado
	\item[JSON] Notação de Objetos JavaScript
	\item[SQL] Linguagem de Consulta Estrurada
	\item[UCP] Unidade Central de Processamento
	
\end{siglas}


% --- Lista de Símbolos (Opcional) ---
% [A FAZER: Descomente se você tiver uma lista de símbolos]
% \begin{simbolos}
%    \item[$ \alpha $] -- Descrição do símbolo alfa.
%    \item[$ \beta $] -- Descrição do símbolo beta.
% \end{simbolos}
% \cleardoublepage

% --- Sumário ---
\pdfbookmark[0]{\contentsname}{toc}
\tableofcontents*
\cleardoublepage

% ----------------------------------------------------------
% -- ELEMENTOS TEXTUAIS (CAPÍTULOS)
% ----------------------------------------------------------
\textual

% [A FAZER: Escreva seus capítulos nestes arquivos .tex separados]
% Esta abordagem modular é altamente recomendada para gerenciar documentos longos.
\chapter{Introdução}\label{cap:intro}

\section{Contextualização}
A disseminação de conhecimento acadêmico encontrou na web seu principal canal de expansão. No entanto, para docentes e pesquisadores — especialmente de áreas não tecnológicas —, a criação de um espaço próprio para publicação de artigos e materiais didáticos ainda esbarra em um dilema: a dependência de plataformas proprietárias e rígidas ou a complexidade de gerenciar infraestruturas de sistemas de gerenciamento de conteúdo (\textit{Content Management Systems} - CMS) tradicionais.

Muitos educadores recorrem a redes sociais ou plataformas de terceiros para compartilhar seus estudos, o que acarreta na perda de controle sobre a identidade visual, na dificuldade de formatação de elementos técnicos — como fórmulas matemáticas e blocos de código — e na fragilidade quanto à perenidade do conteúdo. Nesse cenário, surge o interesse em desenvolver uma ferramenta que una a autonomia de um domínio próprio com a simplicidade de um serviço pronto para uso, reduzindo as barreiras tecnológicas.

\section{Motivação}
O projeto \textbf{Vitrine} nasce da convicção de que cada educador deve ter a capacidade de possuir sua própria "vitrine" de conhecimento, de forma independente e personalizada. A motivação principal reside em transformar o processo de implantação (\textit{deployment}) em um ato acessível, permitindo que a barreira técnica não seja um impedimento para a presença digital qualificada do docente. 

Ao oferecer uma solução \textit{open source}, modular e versátil, o projeto busca devolver ao autor o controle total sobre sua produção intelectual e a estética de sua apresentação. Aprender e compartilhar conhecimento são experiências transformadoras e, conforme destacado por \cite{KeRainerJr2019}, sistemas de informação eficazes devem servir como facilitadores desse processo, e não como obstáculos.

\section{Objetivos}

\subsection{Objetivo Geral}
Desenvolver o \textbf{Vitrine}, uma plataforma CMS \textit{open source}, modular e de fácil implantação, baseada no framework Next.js, voltada para a criação de blogs e portfólios acadêmicos por professores e pesquisadores.

\subsection{Objetivos Específicos}
\begin{itemize}
    \item Viabilizar a publicação independente por meio de um processo de \textit{clonagem} e \textit{deployment} simplificado;
    \item Implementar um editor de texto baseado em \textit{markdown} que suporte necessidades da escrita científica, como realce de sintaxe e fórmulas em \textit{LaTeX};
    \item Prover um sistema de personalização visual dinâmica que permita a alteração da paleta de cores via interface administrativa, sem necessidade de novos ciclos de compilação (\textit{build});
    \item Adotar uma arquitetura de dados portátil utilizando o ORM Prisma e o banco de dados SQLite;
    \item Integrar um sistema de autenticação simplificado através do Auth.js para garantir a segurança no gerenciamento de conteúdos.
\end{itemize}

\section{Justificativa}
A proposta deste trabalho se justifica pela busca do equilíbrio entre modernidade tecnológica e portabilidade. O uso do \textbf{Next.js} permite uma renderização híbrida de alto desempenho, essencial para a indexação em motores de busca (SEO) e para a experiência do usuário. A escolha do \textbf{SQLite} como banco de dados é estratégica: por ser um sistema baseado em arquivo, ele simplifica drasticamente a infraestrutura necessária, reduzindo custos e complexidade de manutenção para o usuário final.

Ao contrário de ambientes virtuais de aprendizagem amplamente utilizados, como o \textbf{Moodle} — cuja proposta central é gerenciar cursos, avaliações e atividades acadêmicas institucionais —, o \textbf{Vitrine} foi concebido especificamente como uma plataforma de publicação autônoma. Embora o Moodle cumpra seu papel em contextos formais, adaptá-lo para fluxos editoriais leves e personalizados exige modificações complexas e adesão a requisitos técnicos institucionais rígidos.

Adicionalmente, a plataforma foca na democratização do acesso à publicação digital. Embora o \textbf{Vitrine} seja projetado para ser intuitivo para usuários sem conhecimento técnico em desenvolvimento web, como professores de áreas de humanidades ou biológicas, ele também se apresenta como uma solução prática para \textbf{desenvolvedores e professores de tecnologia}. Para este público, a ferramenta oferece uma base sólida, moderna e extensível que elimina o trabalho repetitivo de configurar uma estrutura de blog do zero, permitindo que foquem exclusivamente na produção de conteúdo ou na implementação de novos módulos customizados.

A inovação técnica reside na implementação de uma interface que manipula variáveis CSS em tempo de execução, persistindo-as no banco de dados e refletindo-as instantaneamente através de \textit{Incremental Static Regeneration} (ISR). Esta abordagem une a velocidade de um site estático com a flexibilidade de um sistema dinâmico, garantindo a longevidade e a portabilidade dos materiais produzidos.

\section{Organização da Monografia}
Além deste capítulo, os demais estão organizados da seguinte forma:

\begin{itemize}
    \item Capítulo \ref{cap:refTeorico}: Contém o referencial teórico e as tecnologias utilizadas, como React, Next.js e o paradigma Jamstack;
    \item Capítulo \ref{cap:Metodologia}: Apresenta a metodologia de desenvolvimento e o levantamento de requisitos;
    \item Capítulo \ref{cap:devSistema}: Relata as etapas da construção do sistema Vitrine, decisões de arquitetura e testes realizados;
    \item Capítulo \ref{cap:conclusao}: Apresenta as conclusões, limitações do trabalho e propostas de funcionalidades futuras.
\end{itemize}
\chapter{Referencial Teórico}\label{cap:refTeorico}

% Nesta seção, você deve apresentar os conceitos, teorias e tecnologias
% que formam a base do seu trabalho.
\section{Fundamentação Teórica}

O referencial teórico é a base conceitual que sustenta toda a pesquisa. Nesta seção, é essencial apresentar e explicar as teorias, os conceitos, as técnicas e as tecnologias que são fundamentais para a compreensão do seu trabalho. A exposição deve ser clara e organizada, partindo de conceitos mais gerais para os mais específicos.

O objetivo é demonstrar que você possui domínio sobre o conhecimento existente na sua área de estudo. Utilize fontes confiáveis, como livros, artigos científicos e publicações de conferências, para embasar suas definições. Por exemplo, ao discutir sobre desenvolvimento de software, pode-se abordar temas como Engenharia de Requisitos. Segundo Sommerville, a engenharia de requisitos é o processo de descobrir, analisar, documentar e validar os serviços e as restrições de um sistema.

Subseções podem ser utilizadas para organizar os diferentes tópicos abordados, como mostrado a seguir.

\subsection{Metodologias de Desenvolvimento de Software}

As metodologias de desenvolvimento de software fornecem um conjunto de práticas e processos para guiar a construção de sistemas. Elas podem ser classificadas em duas grandes abordagens: preditivas (ou tradicionais) e adaptativas (ou ágeis).

Modelos tradicionais, como o modelo em cascata, seguem um fluxo sequencial e rigoroso, onde cada fase deve ser concluída antes do início da próxima. Em contraste, metodologias ágeis, como Scrum e Kanban, são iterativas e incrementais, focando na entrega contínua de valor e na adaptação a mudanças.

\subsection{Qualidade de Software}

A qualidade de software é um conceito multidimensional que pode ser avaliado sob diferentes perspectivas. A norma ISO/IEC 25010 define um modelo de qualidade com oito características principais: adequação funcional, eficiência de desempenho, compatibilidade, usabilidade, confiabilidade, segurança, manutenibilidade e portabilidade.

% Nesta seção, analise outros estudos e projetos que já abordaram um problema
% semelhante ao seu.
\section{Trabalhos Relacionados}

A análise de trabalhos relacionados é fundamental para situar sua pesquisa no contexto científico atual. Nesta seção, você deve apresentar e discutir outros estudos, artigos ou projetos que abordaram problemas semelhantes ou que utilizaram abordagens relevantes para o seu trabalho.

O objetivo não é apenas listar o que outros autores fizeram, mas sim realizar uma análise crítica, destacando:
\begin{itemize}
    \item As contribuições de cada trabalho.
    \item As metodologias utilizadas.
    \item Os resultados alcançados.
    \item As limitações ou lacunas deixadas por esses estudos, que justificam a sua pesquisa.
\end{itemize}

Por exemplo: "O trabalho de Silva propôs um sistema para otimização de rotas utilizando algoritmos genéticos, alcançando uma redução de 15\% nos custos logísticos. No entanto, a solução não considera restrições de tempo real, uma lacuna que este trabalho busca preencher."

Esta análise crítica demonstra que sua pesquisa é original e contribui de forma significativa para a área, avançando em relação ao que já foi feito.
\chapter{Metodologia}\label{cap:Metodologia}

Neste capítulo, detalha-se a metodologia de pesquisa e o processo de engenharia de software empregados para o desenvolvimento do sistema Vitrine. A abordagem adotada visa garantir o rigor científico necessário para um trabalho acadêmico de Ciência da Computação, aliando a fundamentação teórica à aplicação prática de tecnologias modernas.

\section{Classificação da Pesquisa}

Quanto à sua \textbf{natureza}, esta pesquisa é classificada como \textbf{aplicada}, uma vez que objetiva a criação de um artefato tecnológico — um sistema de gerenciamento de conteúdo (CMS) — destinado a resolver o problema prático da publicação acadêmica independente.

Quanto aos \textbf{objetivos}, o trabalho possui caráter \textbf{descritivo} e \textbf{exploratório}. É exploratório na fase de investigação de novas tecnologias, como o framework Next.js e o paradigma Jamstack, e descritivo ao detalhar as etapas de modelagem, arquitetura e os requisitos funcionais que compõem a solução proposta.

\section{Procedimentos Técnicos}

Para a condução do estudo, foram adotados os seguintes procedimentos:

\subsection{Revisão Bibliográfica}
Realizou-se uma pesquisa sistemática em bases de dados como Google Scholar e documentações técnicas oficiais. O foco recaiu sobre arquiteturas de software para a web, sistemas de bancos de dados portáteis (SQLite) e padrões de design para interfaces de usuário (UI), fornecendo o embasamento necessário para as tomadas de decisão arquiteturais do projeto.

\subsection{Desenvolvimento Experimental e Incremental}
O método principal baseia-se no desenvolvimento de um protótipo funcional através de um processo \textbf{iterativo e incremental}. Diferente de modelos rígidos, esta abordagem permite que as funcionalidades sejam levantadas, modeladas e implementadas em ciclos, garantindo que a complexidade da estrutura hierárquica e da personalização dinâmica de cores seja validada de ponta a ponta a cada incremento.

\section{Etapas do Desenvolvimento}

O processo de construção do Vitrine foi estruturado em quatro macroetapas, integrando a análise acadêmica com as práticas de mercado:

\begin{enumerate}
    \item \textbf{Levantamento de Requisitos:} Identificação das necessidades dos usuários (professores e desenvolvedores), resultando na definição de Requisitos Funcionais (RF) e Não-Funcionais (RNF).
    \item \textbf{Modelagem de Sistemas:} Elaboração de diagramas utilizando a linguagem UML e DBML para visualização da lógica do sistema:
    \begin{itemize}
        \item \textbf{Diagrama de Casos de Uso:} Para mapear as interações entre os atores (Leitor, Administrador e Sistema) e as funcionalidades principais.
        \item \textbf{Modelagem de Dados (ER):} Definição da estrutura do banco de dados SQLite, com foco na auto-relação para suportar a hierarquia de conteúdos.
    \end{itemize}
    \item \textbf{Prototipagem e Design de Interface:} Criação de protótipos de baixa e alta fidelidade para validar a usabilidade do editor de Markdown e do painel de configuração de cores antes da implementação final.
    \item \textbf{Desenvolvimento Incremental (Sprints):} Implementação do código-fonte utilizando Next.js e Prisma ORM. Cada funcionalidade é desenvolvida de "ponta a ponta", abrangendo desde a criação da tabela no banco de dados até a renderização no front-end e configuração de cache (ISR).
\end{enumerate}



\section{Cronograma de Desenvolvimento}

O cronograma abaixo detalha a distribuição das atividades ao longo do período de graduação, enfatizando o caráter incremental da implementação.

\begin{table}[h!]
\centering
\caption{Cronograma de Execução do Projeto Vitrine}
\label{tab:cronograma}
\renewcommand{\arraystretch}{1.3}
\begin{tabular}{@{}p{3.5cm} p{5.5cm} p{2.5cm} p{2.5cm}@{}}
\toprule
\textbf{Etapa} & \textbf{Atividade Específica} & \textbf{Início} & \textbf{Término} \\
\midrule
\multirow{2}{*}{Concepção}
& Levantamento de Requisitos e Escopo & Mês 1 & Mês 1 \\
& Revisão Bibliográfica e Tecnológica & Mês 1 & Mês 2 \\
\midrule
\multirow{3}{*}{Modelagem}
& Diagramas de Casos de Uso e ER & Mês 2 & Mês 2 \\
& Prototipagem da Interface (UI/UX) & Mês 2 & Mês 3 \\
& Definição da Arquitetura e Stack & Mês 3 & Mês 3 \\
\midrule
\multirow{3}{*}{Desenvolvimento}
& Setup do Ambiente e Auth.js & Mês 3 & Mês 4 \\
& Módulo de Conteúdo e Markdown & Mês 4 & Mês 6 \\
& Módulo de Personalização e ISR & Mês 6 & Mês 7 \\
\midrule
\multirow{2}{*}{Finalização}
& Escrita da Monografia e Testes & Mês 4 & Mês 8 \\
& Preparação da Defesa & Mês 8 & Mês 8 \\
\bottomrule
\end{tabular}
\end{table}

\section{Requisitos do Sistema}
A especificação dos requisitos foi a base para o desenvolvimento incremental, dividindo-se em necessidades funcionais e restrições técnicas.

\subsection{Requisitos Funcionais (RF)}
\begin{itemize}
    \item \textbf{RF01 - Autenticação:} Acesso restrito via Auth.js.
    \item \textbf{RF02 - Estrutura Hierárquica:} Suporte a documentos e coleções aninhadas.
    \item \textbf{RF03 - Ordenação:} Atributo de ordem relativa entre elementos irmãos.
    \item \textbf{RF04 - Personalização:} Interface para alteração de branding e cores (CSS Variables).
\end{itemize}

\subsection{Diagrama de Casos de Uso}
O diagrama de casos de uso (Figura \ref{fig:uc}) ilustra as interações entre o Administrador, o Leitor e as automações de sistema (ISR).

\begin{figure}[h!]
    \centering
    \includegraphics[width=0.9\textwidth]{imgs/use-case-diagram.png}
    \caption{Diagrama de Casos de Uso do sistema Vitrine.}
    \label{fig:uc}
\end{figure}
\chapter{Desenvolvimento}\label{cap:desenvolvimento}

Neste capítulo, detalha-se o processo de construção técnica do sistema Vitrine. A implementação foi guiada pela busca de uma solução que aliasse a robustez de frameworks modernos à simplicidade de manutenção exigida pelo público-alvo de educadores e pesquisadores.

\section{Desenvolvimento da Solução Proposta}
O desenvolvimento do Vitrine foi pautado na criação de um CMS modular. Seguindo a metodologia incremental, o sistema foi construído sobre o ecossistema \textit{fullstack} do framework Next.js, utilizando o paradigma de \textit{App Router} para otimizar a renderização e o roteamento.

\subsection{Arquitetura do Sistema}
A arquitetura do sistema foi projetada para ser desacoplada e portátil, fundamentada no padrão de renderização híbrida. O sistema divide-se em três camadas principais:

\begin{itemize}
    \item \textbf{Camada de Apresentação (Front-end):} Desenvolvida com React e Tailwind CSS, focada em entregar uma interface responsiva e componentes modulares que consomem variáveis CSS dinâmicas para a personalização visual.
    \item \textbf{Camada de Aplicação (Back-end):} Composta por \textit{Server Components} e \textit{Server Actions} do Next.js, eliminando a necessidade de uma API externa separada e simplificando a lógica de negócio e autenticação via Auth.js.
    \item \textbf{Camada de Persistência:} Utiliza o banco de dados SQLite gerenciado pelo Prisma ORM. Esta camada é responsável por armazenar tanto o conteúdo em Markdown quanto as configurações de identidade visual do sistema.
\end{itemize}

\subsection{Modelagem de Dados}
A persistência de dados foi modelada para garantir a portabilidade no SQLite. O Diagrama de Entidade-Relacionamento (Figura \ref{fig:er}) destaca a tabela \textit{contents} e sua natureza recursiva, permitindo representar estruturas de dados em árvore de profundidade arbitrária. Para a recuperação eficiente desses dados, implementou-se uma lógica de ordenação baseada no atributo \texttt{order}, garantindo que elementos no mesmo nível da hierarquia mantenham a sequência definida pelo administrador.

\begin{figure}[h!]
    \centering
    \includegraphics[width=0.8\textwidth]{imgs/entity-relationship-diagram.png}
    \caption{Diagrama de Entidade-Relacionamento com foco na auto-relação.}
    \label{fig:er}
\end{figure}

\subsection{Prototipagem de Alta Fidelidade}
Antes da codificação final, foram desenvolvidos protótipos de alta fidelidade para validar a experiência do usuário (UX) e a interface (UI). Estes protótipos serviram como guia para a implementação dos componentes React e para a definição das variáveis de estilo dinâmicas.

A Figura \ref{fig:proto_admin} apresenta a interface de gerenciamento, onde é possível observar a disposição da barra lateral hierárquica e o editor de Markdown. A Figura \ref{fig:proto_style} ilustra o módulo de personalização, evidenciando como os seletores de cores interagem com a pré-visualização do sistema.

\begin{figure}[h!]
    \centering
    \includegraphics[width=0.8\textwidth]{imgs/university-logo.jpg}
    \caption{Protótipo de alta fidelidade do painel administrativo e editor.}
    \label{fig:proto_admin}
\end{figure}

\begin{figure}[h!]
    \centering
    \includegraphics[width=0.8\textwidth]{imgs/university-logo.jpg}
    \caption{Protótipo da interface de personalização de cores e branding.}
    \label{fig:proto_style}
\end{figure}

\subsection{Principais Funcionalidades Implementadas}
As funcionalidades implementadas visam reduzir a carga cognitiva do usuário final, automatizando processos complexos de desenvolvimento web.

\subsubsection{Personalização de UI com Variáveis Dinâmicas}
O sistema lê os valores hexadecimais salvos na tabela de configurações e os injeta como variáveis de ambiente CSS (\textit{CSS Variables}) no componente raiz da aplicação. Para manter a performance, utiliza-se o mecanismo de \textit{Incremental Static Regeneration} (ISR), que revalida o cache da página no momento em que as configurações são salvas.

\subsubsection{Editor de Markdown e Renderização Científica}
O sistema integra um editor de texto otimizado que persiste dados em Markdown. A renderização no front-end suporta realce de sintaxe para múltiplas linguagens de programação e a compilação de expressões matemáticas escritas em \textit{LaTeX}.

\subsubsection{Sistema de Gestão de Conteúdo (Admin Panel)}
O painel administrativo permite a manipulação da estrutura de arquivos do blog de forma visual. O administrador pode alternar entre tipos de conteúdo (\textit{Document} ou \textit{Collection}), reordenar itens e gerenciar o \textit{deploy} de novos materiais de forma intuitiva.
\chapter{Resultados e Discussão}\label{cap:resultados}
% O nome original do arquivo foi mantido, mas o título do capítulo foi alterado
% para ser mais genérico e adequado a um template.

Neste capítulo, são apresentados os resultados obtidos a partir da aplicação da metodologia descrita no Capítulo \ref{cap:Metodologia}. A apresentação é seguida por uma discussão que interpreta esses resultados, correlacionando-os com o referencial teórico e os objetivos do trabalho.

\section{Apresentação do Artefato Desenvolvido}

% [A FAZER: Se o seu trabalho envolveu o desenvolvimento de um software, protótipo
% ou outro artefato, descreva-o aqui.]

Como resultado principal desta pesquisa, foi desenvolvido um protótipo de software denominado [Nome do Sistema]. O sistema foi projetado para [descrever o objetivo principal do sistema]. A arquitetura do sistema, conforme projetada, é composta por um \textit{front-end} desenvolvido em React e um \textit{back-end} em Node.js, comunicando-se através de uma API RESTful.

As principais funcionalidades implementadas são:
\begin{itemize}
    \item \textbf{Cadastro de Usuários:} Permite que novos usuários se registrem na plataforma.
    \item \textbf{Módulo de [Funcionalidade 1]:} Descreve a primeira funcionalidade principal, como ela funciona e qual problema resolve.
    \item \textbf{Módulo de [Funcionalidade 2]:} Descreve a segunda funcionalidade principal.
\end{itemize}

A Figura \ref{fig:tela_principal} ilustra a tela principal do sistema.

% \begin{figure}[h]
%     \centering
%     % [A FAZER: Substitua 'placeholder.jpg' por uma imagem da sua solução]
%     \includegraphics[width=0.8\textwidth]{imgs/placeholder.jpg}
%     \caption{Tela principal do sistema [Nome do Sistema].}
%     \label{fig:tela_principal}
% \end{figure}

\section{Análise dos Resultados da Avaliação}

% [A FAZER: Apresente os dados coletados na fase de validação/testes.
% Use tabelas, gráficos e outras formas de visualização de dados.]

Para validar a solução proposta, foram realizados testes de usabilidade com um grupo de [número] usuários. Os participantes foram convidados a executar um conjunto de tarefas pré-definidas no sistema, e seus resultados foram coletados por meio de questionários e observação direta.

A Tabela \ref{tab:resultados_testes} resume os resultados obtidos a partir do questionário de satisfação, que utilizou uma escala Likert de 1 (discordo totalmente) a 5 (concordo totalmente).

\begin{table}[h!]
\centering
\caption{Resultados do Questionário de Satisfação do Usuário.}
\label{tab:resultados_testes}
\begin{tabular}{@{}lc@{}}
\toprule
\textbf{Afirmação} & \textbf{Média das Respostas} \\
\midrule
O sistema é fácil de usar. & 4.5 \\
As funcionalidades são úteis para a tarefa. & 4.8 \\
Eu me senti satisfeito(a) ao usar o sistema. & 4.6 \\
\bottomrule
\end{tabular}
\end{table}

\section{Discussão}

% [A FAZER: Interprete os resultados apresentados. O que eles significam?
% Eles confirmam suas hipóteses? Compare-os com os trabalhos relacionados.]

Os resultados apresentados na seção anterior indicam uma alta aceitação do protótipo pelos usuários. A média de 4.5 na afirmação "O sistema é fácil de usar" sugere que a interface proposta é intuitiva e atende às expectativas de usabilidade.

Comparando esses resultados com o trabalho de Fulano, que desenvolveu uma ferramenta similar, nota-se que nossa solução obteve um índice de satisfação superior. Isso pode ser atribuído à [explique uma possível causa, ex: adoção de um design mais limpo ou a inclusão de uma funcionalidade chave].

Apesar dos resultados positivos, foi identificado durante os testes que [descreva uma limitação ou ponto de melhoria]. Este ponto é discutido como uma limitação do trabalho na Conclusão e serve como sugestão para trabalhos futuros. Os resultados alcançados confirmam que os objetivos propostos para este trabalho foram atingidos com sucesso.
\chapter{Conclusão}\label{cap:conclusao}

Este capítulo finaliza o trabalho apresentando as considerações finais sobre a pesquisa realizada, as principais contribuições, as limitações identificadas e as sugestões para trabalhos futuros. O objetivo é consolidar o conhecimento gerado e indicar novos caminhos de investigação a partir dos resultados obtidos.

\section{Considerações Finais e Contribuições}

O presente trabalho teve como objetivo [retome o objetivo geral de forma sucinta]. Para alcançá-lo, foi realizada uma extensa revisão da literatura sobre [tópico 1] e [tópico 2], seguida pelo desenvolvimento de [descreva o artefato ou resultado principal].

Os resultados, apresentados no Capítulo \ref{cap:devSistema}, demonstraram que [resuma a principal conclusão do trabalho]. Isso confirma a hipótese inicial de que [mencione a hipótese, se houver] e atende aos objetivos específicos propostos.

As principais contribuições deste trabalho podem ser resumidas em:
\begin{itemize}
    \item \textbf{Contribuição Teórica:} O aprofundamento no estudo de [conceito teórico], relacionando-o com [outro conceito].
    \item \textbf{Contribuição Prática:} O desenvolvimento de um [protótipo/modelo/framework] que soluciona o problema de [descreva o problema] de forma eficaz, conforme validado pelos testes realizados.
    \item \textbf{Contribuição Acadêmica:} A disponibilização de um novo conjunto de dados (ou uma nova abordagem) que pode servir de base para futuras pesquisas na área.
\end{itemize}

\section{Limitações do Trabalho}\label{sec:limitacoes}

Apesar de os objetivos terem sido alcançados, é importante reconhecer as limitações inerentes a esta pesquisa. Toda investigação possui um escopo delimitado, e a consciência de suas fronteiras é um sinal de maturidade científica.

Uma das principais limitações foi [descreva uma limitação, por exemplo, o tamanho da amostra utilizada nos testes]. O grupo de validação foi composto por [número] participantes, o que, embora suficiente para uma avaliação inicial, pode não representar a totalidade do público-alvo.

Outra limitação refere-se ao [descreva outra limitação, por exemplo, o escopo das funcionalidades implementadas]. O protótipo desenvolvido contempla as funcionalidades essenciais, mas funcionalidades adicionais, como [exemplo de funcionalidade], não foram implementadas devido a restrições de tempo.

\section{Trabalhos Futuros}

Com base nos resultados e nas limitações identificadas, diversas oportunidades para a continuidade e aprofundamento desta pesquisa podem ser sugeridas. As propostas de trabalhos futuros incluem:

\begin{itemize}
    \item \textbf{Expandir a Validação:} Realizar uma nova rodada de testes com um grupo maior e mais diversificado de usuários para obter resultados estatisticamente mais robustos.
    \item \textbf{Implementar Novas Funcionalidades:} Desenvolver as funcionalidades que não foram incluídas no escopo deste trabalho, como [exemplo de funcionalidade 1] e [exemplo de funcionalidade 2], para tornar a solução mais completa.
    \item \textbf{Explorar Outras Tecnologias:} Investigar a aplicação de [outra tecnologia, ex: aprendizado de máquina] para aprimorar o [aspecto do sistema], visando [objetivo, ex: automatizar a tomada de decisão].
    \item \textbf{Aplicar a Outros Contextos:} Adaptar e avaliar a solução proposta em diferentes domínios ou contextos de aplicação para verificar sua generalização.
\end{itemize}

Essas sugestões demonstram o potencial de continuidade da pesquisa e abrem novas frentes de investigação para a comunidade acadêmica.

% ----------------------------------------------------------
% -- ELEMENTOS PÓS-TEXTUAIS
% ----------------------------------------------------------
\postextual
\bookmarksetup{startatroot} % Reseta a hierarquia dos bookmarks para os elementos pós-textuais

% --- Referências ---
% O estilo 'abnt-num' cria citações numéricas. Para autor-data, use 'abnt-alf'.
\bibliographystyle{abnt-num}
% O arquivo .bib contendo suas referências
% [A FAZER: Adicione suas referências ao arquivo 'references.bib']
\bibliography{references}

% --- Apêndices (Opcional) ---
% [A FAZER: Descomente se você tiver apêndices]
% \appendix
% %% ===================================================================================
%% Apêndices e Anexos (appendix_a.tex)
%% Versão: 2.2 (Corrigida)
%%
%% Este arquivo contém os elementos pós-textuais do seu trabalho. Ele separa
%% claramente Apêndices (trabalho do próprio autor) de Anexos (material de terceiros).
%%
%% Para usar este arquivo:
%% 1. Adicione seu conteúdo sob as seções apropriadas abaixo.
%% 2. No arquivo main.tex, descomente os blocos \appendix ou \anexos e use
%%    %% ===================================================================================
%% Apêndices e Anexos (appendix_a.tex)
%% Versão: 2.2 (Corrigida)
%%
%% Este arquivo contém os elementos pós-textuais do seu trabalho. Ele separa
%% claramente Apêndices (trabalho do próprio autor) de Anexos (material de terceiros).
%%
%% Para usar este arquivo:
%% 1. Adicione seu conteúdo sob as seções apropriadas abaixo.
%% 2. No arquivo main.tex, descomente os blocos \appendix ou \anexos e use
%%    %% ===================================================================================
%% Apêndices e Anexos (appendix_a.tex)
%% Versão: 2.2 (Corrigida)
%%
%% Este arquivo contém os elementos pós-textuais do seu trabalho. Ele separa
%% claramente Apêndices (trabalho do próprio autor) de Anexos (material de terceiros).
%%
%% Para usar este arquivo:
%% 1. Adicione seu conteúdo sob as seções apropriadas abaixo.
%% 2. No arquivo main.tex, descomente os blocos \appendix ou \anexos e use
%%    \include{posttextual/appendix_a} para adicionar este conteúdo ao seu documento.
%% ===================================================================================

% -----------------------------------------------------------------------------------
% -- APÊNDICES ----------------------------------------------------------------------
% Um Apêndice contém material suplementar CRIADO PELO AUTOR.
% Exemplos: código-fonte, dados de pesquisa, provas matemáticas detalhadas.
% -----------------------------------------------------------------------------------
% \appendix % Este comando inicia a seção de apêndices, numerando os capítulos com letras (A, B, C...).

% [A FAZER: Adicione seus apêndices abaixo. Cada apêndice deve ser um novo \chapter.]
% Se você não tiver apêndices, pode deletar esta seção.

\chapter{Código Fonte do Módulo de Autenticação}
\label{ap:codigo_fonte}

Este apêndice contém o código-fonte completo para o módulo de autenticação discutido no Capítulo 4. O código é fornecido para referência e completude.

Você pode adicionar blocos de código com realce de sintaxe usando o ambiente `minted`.
\begin{minted}[linenos, frame=lines]{python}
# auth_module.py

def validar_usuario(username, password):
    """
    Valida as credenciais do usuário contra o banco de dados.
    Esta é uma função de exemplo.
    """
    if not username or not password:
        return False
    # Em uma aplicação real, você consultaria um banco de dados aqui.
    print(f"Validando usuário: {username}")
    return True
\end{minted}


% -----------------------------------------------------------------------------------
% -- ANEXOS -------------------------------------------------------------------------
% Um Anexo contém material de apoio DE OUTRAS FONTES.
% Exemplos: mapas, leis, manuais técnicos de um fabricante.
% -----------------------------------------------------------------------------------
% \anexos % Este comando inicia a seção de anexos, continuando a numeração com letras.

% [A FAZER: Adicione seus anexos abaixo. Cada anexo deve ser um novo \chapter.]
% Se você não tiver anexos, pode deletar esta seção.

\chapter{Manual Técnico do Sensor Y}
\label{anx:manual}

Este anexo fornece o manual técnico oficial do fabricante para o sensor de temperatura utilizado no experimento (Capítulo 3). O documento especifica sua faixa de operação e precisão.

Você pode incluir um documento externo (como um PDF ou imagem) como uma figura.
\begin{figure}[H] % A opção [H] do pacote 'float' posiciona a figura exatamente aqui.
    \centering
    % \includegraphics[width=0.8\textwidth]{caminho/para/imagem_manual.png}
    \fbox{Espaço reservado para uma imagem ou documento escaneado.}
    \caption{Especificações técnicas do Sensor Y.} para adicionar este conteúdo ao seu documento.
%% ===================================================================================

% -----------------------------------------------------------------------------------
% -- APÊNDICES ----------------------------------------------------------------------
% Um Apêndice contém material suplementar CRIADO PELO AUTOR.
% Exemplos: código-fonte, dados de pesquisa, provas matemáticas detalhadas.
% -----------------------------------------------------------------------------------
% \appendix % Este comando inicia a seção de apêndices, numerando os capítulos com letras (A, B, C...).

% [A FAZER: Adicione seus apêndices abaixo. Cada apêndice deve ser um novo \chapter.]
% Se você não tiver apêndices, pode deletar esta seção.

\chapter{Código Fonte do Módulo de Autenticação}
\label{ap:codigo_fonte}

Este apêndice contém o código-fonte completo para o módulo de autenticação discutido no Capítulo 4. O código é fornecido para referência e completude.

Você pode adicionar blocos de código com realce de sintaxe usando o ambiente `minted`.
\begin{minted}[linenos, frame=lines]{python}
# auth_module.py

def validar_usuario(username, password):
    """
    Valida as credenciais do usuário contra o banco de dados.
    Esta é uma função de exemplo.
    """
    if not username or not password:
        return False
    # Em uma aplicação real, você consultaria um banco de dados aqui.
    print(f"Validando usuário: {username}")
    return True
\end{minted}


% -----------------------------------------------------------------------------------
% -- ANEXOS -------------------------------------------------------------------------
% Um Anexo contém material de apoio DE OUTRAS FONTES.
% Exemplos: mapas, leis, manuais técnicos de um fabricante.
% -----------------------------------------------------------------------------------
% \anexos % Este comando inicia a seção de anexos, continuando a numeração com letras.

% [A FAZER: Adicione seus anexos abaixo. Cada anexo deve ser um novo \chapter.]
% Se você não tiver anexos, pode deletar esta seção.

\chapter{Manual Técnico do Sensor Y}
\label{anx:manual}

Este anexo fornece o manual técnico oficial do fabricante para o sensor de temperatura utilizado no experimento (Capítulo 3). O documento especifica sua faixa de operação e precisão.

Você pode incluir um documento externo (como um PDF ou imagem) como uma figura.
\begin{figure}[H] % A opção [H] do pacote 'float' posiciona a figura exatamente aqui.
    \centering
    % \includegraphics[width=0.8\textwidth]{caminho/para/imagem_manual.png}
    \fbox{Espaço reservado para uma imagem ou documento escaneado.}
    \caption{Especificações técnicas do Sensor Y.} para adicionar este conteúdo ao seu documento.
%% ===================================================================================

% -----------------------------------------------------------------------------------
% -- APÊNDICES ----------------------------------------------------------------------
% Um Apêndice contém material suplementar CRIADO PELO AUTOR.
% Exemplos: código-fonte, dados de pesquisa, provas matemáticas detalhadas.
% -----------------------------------------------------------------------------------
% \appendix % Este comando inicia a seção de apêndices, numerando os capítulos com letras (A, B, C...).

% [A FAZER: Adicione seus apêndices abaixo. Cada apêndice deve ser um novo \chapter.]
% Se você não tiver apêndices, pode deletar esta seção.

\chapter{Código Fonte do Módulo de Autenticação}
\label{ap:codigo_fonte}

Este apêndice contém o código-fonte completo para o módulo de autenticação discutido no Capítulo 4. O código é fornecido para referência e completude.

Você pode adicionar blocos de código com realce de sintaxe usando o ambiente `minted`.
\begin{minted}[linenos, frame=lines]{python}
# auth_module.py

def validar_usuario(username, password):
    """
    Valida as credenciais do usuário contra o banco de dados.
    Esta é uma função de exemplo.
    """
    if not username or not password:
        return False
    # Em uma aplicação real, você consultaria um banco de dados aqui.
    print(f"Validando usuário: {username}")
    return True
\end{minted}


% -----------------------------------------------------------------------------------
% -- ANEXOS -------------------------------------------------------------------------
% Um Anexo contém material de apoio DE OUTRAS FONTES.
% Exemplos: mapas, leis, manuais técnicos de um fabricante.
% -----------------------------------------------------------------------------------
% \anexos % Este comando inicia a seção de anexos, continuando a numeração com letras.

% [A FAZER: Adicione seus anexos abaixo. Cada anexo deve ser um novo \chapter.]
% Se você não tiver anexos, pode deletar esta seção.

\chapter{Manual Técnico do Sensor Y}
\label{anx:manual}

Este anexo fornece o manual técnico oficial do fabricante para o sensor de temperatura utilizado no experimento (Capítulo 3). O documento especifica sua faixa de operação e precisão.

Você pode incluir um documento externo (como um PDF ou imagem) como uma figura.
\begin{figure}[H] % A opção [H] do pacote 'float' posiciona a figura exatamente aqui.
    \centering
    % \includegraphics[width=0.8\textwidth]{caminho/para/imagem_manual.png}
    \fbox{Espaço reservado para uma imagem ou documento escaneado.}
    \caption{Especificações técnicas do Sensor Y.}

% --- Anexos (Opcional) ---
% [A FAZER: Descomente se você tiver anexos]
% \anexos
% \include{posttextual/attachment_a}

% --- Índice Remissivo (Opcional) ---
% [A FAZER: Descomente para imprimir o índice remissivo]
% \printindex

\end{document}