\chapter{Conclusão}\label{cap:conclusao}

Este capítulo finaliza o trabalho apresentando as considerações finais sobre a pesquisa realizada, as principais contribuições, as limitações identificadas e as sugestões para trabalhos futuros. O objetivo é consolidar o conhecimento gerado e indicar novos caminhos de investigação a partir dos resultados obtidos.

\section{Considerações Finais e Contribuições}

O presente trabalho teve como objetivo [retome o objetivo geral de forma sucinta]. Para alcançá-lo, foi realizada uma extensa revisão da literatura sobre [tópico 1] e [tópico 2], seguida pelo desenvolvimento de [descreva o artefato ou resultado principal].

Os resultados, apresentados no Capítulo \ref{cap:devSistema}, demonstraram que [resuma a principal conclusão do trabalho]. Isso confirma a hipótese inicial de que [mencione a hipótese, se houver] e atende aos objetivos específicos propostos.

As principais contribuições deste trabalho podem ser resumidas em:
\begin{itemize}
    \item \textbf{Contribuição Teórica:} O aprofundamento no estudo de [conceito teórico], relacionando-o com [outro conceito].
    \item \textbf{Contribuição Prática:} O desenvolvimento de um [protótipo/modelo/framework] que soluciona o problema de [descreva o problema] de forma eficaz, conforme validado pelos testes realizados.
    \item \textbf{Contribuição Acadêmica:} A disponibilização de um novo conjunto de dados (ou uma nova abordagem) que pode servir de base para futuras pesquisas na área.
\end{itemize}

\section{Limitações do Trabalho}\label{sec:limitacoes}

Apesar de os objetivos terem sido alcançados, é importante reconhecer as limitações inerentes a esta pesquisa. Toda investigação possui um escopo delimitado, e a consciência de suas fronteiras é um sinal de maturidade científica.

Uma das principais limitações foi [descreva uma limitação, por exemplo, o tamanho da amostra utilizada nos testes]. O grupo de validação foi composto por [número] participantes, o que, embora suficiente para uma avaliação inicial, pode não representar a totalidade do público-alvo.

Outra limitação refere-se ao [descreva outra limitação, por exemplo, o escopo das funcionalidades implementadas]. O protótipo desenvolvido contempla as funcionalidades essenciais, mas funcionalidades adicionais, como [exemplo de funcionalidade], não foram implementadas devido a restrições de tempo.

\section{Trabalhos Futuros}

Com base nos resultados e nas limitações identificadas, diversas oportunidades para a continuidade e aprofundamento desta pesquisa podem ser sugeridas. As propostas de trabalhos futuros incluem:

\begin{itemize}
    \item \textbf{Expandir a Validação:} Realizar uma nova rodada de testes com um grupo maior e mais diversificado de usuários para obter resultados estatisticamente mais robustos.
    \item \textbf{Implementar Novas Funcionalidades:} Desenvolver as funcionalidades que não foram incluídas no escopo deste trabalho, como [exemplo de funcionalidade 1] e [exemplo de funcionalidade 2], para tornar a solução mais completa.
    \item \textbf{Explorar Outras Tecnologias:} Investigar a aplicação de [outra tecnologia, ex: aprendizado de máquina] para aprimorar o [aspecto do sistema], visando [objetivo, ex: automatizar a tomada de decisão].
    \item \textbf{Aplicar a Outros Contextos:} Adaptar e avaliar a solução proposta em diferentes domínios ou contextos de aplicação para verificar sua generalização.
\end{itemize}

Essas sugestões demonstram o potencial de continuidade da pesquisa e abrem novas frentes de investigação para a comunidade acadêmica.