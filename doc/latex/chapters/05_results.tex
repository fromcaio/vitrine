\chapter{Resultados e Discussão}\label{cap:resultados}
% O nome original do arquivo foi mantido, mas o título do capítulo foi alterado
% para ser mais genérico e adequado a um template.

Neste capítulo, são apresentados os resultados obtidos a partir da aplicação da metodologia descrita no Capítulo \ref{cap:Metodologia}. A apresentação é seguida por uma discussão que interpreta esses resultados, correlacionando-os com o referencial teórico e os objetivos do trabalho.

\section{Apresentação do Artefato Desenvolvido}

% [A FAZER: Se o seu trabalho envolveu o desenvolvimento de um software, protótipo
% ou outro artefato, descreva-o aqui.]

Como resultado principal desta pesquisa, foi desenvolvido um protótipo de software denominado [Nome do Sistema]. O sistema foi projetado para [descrever o objetivo principal do sistema]. A arquitetura do sistema, conforme projetada, é composta por um \textit{front-end} desenvolvido em React e um \textit{back-end} em Node.js, comunicando-se através de uma API RESTful.

As principais funcionalidades implementadas são:
\begin{itemize}
    \item \textbf{Cadastro de Usuários:} Permite que novos usuários se registrem na plataforma.
    \item \textbf{Módulo de [Funcionalidade 1]:} Descreve a primeira funcionalidade principal, como ela funciona e qual problema resolve.
    \item \textbf{Módulo de [Funcionalidade 2]:} Descreve a segunda funcionalidade principal.
\end{itemize}

A Figura \ref{fig:tela_principal} ilustra a tela principal do sistema.

% \begin{figure}[h]
%     \centering
%     % [A FAZER: Substitua 'placeholder.jpg' por uma imagem da sua solução]
%     \includegraphics[width=0.8\textwidth]{imgs/placeholder.jpg}
%     \caption{Tela principal do sistema [Nome do Sistema].}
%     \label{fig:tela_principal}
% \end{figure}

\section{Análise dos Resultados da Avaliação}

% [A FAZER: Apresente os dados coletados na fase de validação/testes.
% Use tabelas, gráficos e outras formas de visualização de dados.]

Para validar a solução proposta, foram realizados testes de usabilidade com um grupo de [número] usuários. Os participantes foram convidados a executar um conjunto de tarefas pré-definidas no sistema, e seus resultados foram coletados por meio de questionários e observação direta.

A Tabela \ref{tab:resultados_testes} resume os resultados obtidos a partir do questionário de satisfação, que utilizou uma escala Likert de 1 (discordo totalmente) a 5 (concordo totalmente).

\begin{table}[h!]
\centering
\caption{Resultados do Questionário de Satisfação do Usuário.}
\label{tab:resultados_testes}
\begin{tabular}{@{}lc@{}}
\toprule
\textbf{Afirmação} & \textbf{Média das Respostas} \\
\midrule
O sistema é fácil de usar. & 4.5 \\
As funcionalidades são úteis para a tarefa. & 4.8 \\
Eu me senti satisfeito(a) ao usar o sistema. & 4.6 \\
\bottomrule
\end{tabular}
\end{table}

\section{Discussão}

% [A FAZER: Interprete os resultados apresentados. O que eles significam?
% Eles confirmam suas hipóteses? Compare-os com os trabalhos relacionados.]

Os resultados apresentados na seção anterior indicam uma alta aceitação do protótipo pelos usuários. A média de 4.5 na afirmação "O sistema é fácil de usar" sugere que a interface proposta é intuitiva e atende às expectativas de usabilidade.

Comparando esses resultados com o trabalho de Fulano, que desenvolveu uma ferramenta similar, nota-se que nossa solução obteve um índice de satisfação superior. Isso pode ser atribuído à [explique uma possível causa, ex: adoção de um design mais limpo ou a inclusão de uma funcionalidade chave].

Apesar dos resultados positivos, foi identificado durante os testes que [descreva uma limitação ou ponto de melhoria]. Este ponto é discutido como uma limitação do trabalho na Conclusão e serve como sugestão para trabalhos futuros. Os resultados alcançados confirmam que os objetivos propostos para este trabalho foram atingidos com sucesso.