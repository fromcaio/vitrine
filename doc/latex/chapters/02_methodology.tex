\chapter{Referencial Teórico}\label{cap:refTeorico}

% Nesta seção, você deve apresentar os conceitos, teorias e tecnologias
% que formam a base do seu trabalho.
\section{Fundamentação Teórica}

O referencial teórico é a base conceitual que sustenta toda a pesquisa. Nesta seção, é essencial apresentar e explicar as teorias, os conceitos, as técnicas e as tecnologias que são fundamentais para a compreensão do seu trabalho. A exposição deve ser clara e organizada, partindo de conceitos mais gerais para os mais específicos.

O objetivo é demonstrar que você possui domínio sobre o conhecimento existente na sua área de estudo. Utilize fontes confiáveis, como livros, artigos científicos e publicações de conferências, para embasar suas definições. Por exemplo, ao discutir sobre desenvolvimento de software, pode-se abordar temas como Engenharia de Requisitos. Segundo Sommerville, a engenharia de requisitos é o processo de descobrir, analisar, documentar e validar os serviços e as restrições de um sistema.

Subseções podem ser utilizadas para organizar os diferentes tópicos abordados, como mostrado a seguir.

\subsection{Metodologias de Desenvolvimento de Software}

As metodologias de desenvolvimento de software fornecem um conjunto de práticas e processos para guiar a construção de sistemas. Elas podem ser classificadas em duas grandes abordagens: preditivas (ou tradicionais) e adaptativas (ou ágeis).

Modelos tradicionais, como o modelo em cascata, seguem um fluxo sequencial e rigoroso, onde cada fase deve ser concluída antes do início da próxima. Em contraste, metodologias ágeis, como Scrum e Kanban, são iterativas e incrementais, focando na entrega contínua de valor e na adaptação a mudanças.

\subsection{Qualidade de Software}

A qualidade de software é um conceito multidimensional que pode ser avaliado sob diferentes perspectivas. A norma ISO/IEC 25010 define um modelo de qualidade com oito características principais: adequação funcional, eficiência de desempenho, compatibilidade, usabilidade, confiabilidade, segurança, manutenibilidade e portabilidade.

% Nesta seção, analise outros estudos e projetos que já abordaram um problema
% semelhante ao seu.
\section{Trabalhos Relacionados}

A análise de trabalhos relacionados é fundamental para situar sua pesquisa no contexto científico atual. Nesta seção, você deve apresentar e discutir outros estudos, artigos ou projetos que abordaram problemas semelhantes ou que utilizaram abordagens relevantes para o seu trabalho.

O objetivo não é apenas listar o que outros autores fizeram, mas sim realizar uma análise crítica, destacando:
\begin{itemize}
    \item As contribuições de cada trabalho.
    \item As metodologias utilizadas.
    \item Os resultados alcançados.
    \item As limitações ou lacunas deixadas por esses estudos, que justificam a sua pesquisa.
\end{itemize}

Por exemplo: "O trabalho de Silva propôs um sistema para otimização de rotas utilizando algoritmos genéticos, alcançando uma redução de 15\% nos custos logísticos. No entanto, a solução não considera restrições de tempo real, uma lacuna que este trabalho busca preencher."

Esta análise crítica demonstra que sua pesquisa é original e contribui de forma significativa para a área, avançando em relação ao que já foi feito.