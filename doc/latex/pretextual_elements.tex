%% ===================================================================================
%% Elementos Pré-Textuais (pretextual_elements.tex)
%% Versão: 2.2 (Corrigida)
%%
%% Este arquivo contém:
%% - Folha de Aprovação
%% - Dedicatória
%% - Agradecimentos
%% - Epígrafe
%% - Resumo em Português
%% - Resumo em Língua Estrangeira (ex: Inglês)
%%
%% Preencha, descomente ou remova as seções conforme sua necessidade.
%% ===================================================================================


% -----------------------------------------------------------------------------------
% -- FOLHA DE APROVAÇÃO
% -----------------------------------------------------------------------------------
% NOTA: Este é um elemento obrigatório (ABNT NBR 14724:2011).
% Você pode usar esta versão em texto para rascunhos. Após a defesa, é uma
% prática comum escanear a página assinada e substituir todo o conteúdo do
% ambiente 'folhadeaprovacao' por: \includegraphics[width=\textwidth]{caminho/para/pagina_assinada.pdf}
% -----------------------------------------------------------------------------------
\begin{folhadeaprovacao}
  \begin{center}
    {\ABNTEXchapterfont\large\imprimirautor}

    \vspace*{\fill}
    {\ABNTEXchapterfont\bfseries\Large\imprimirtitulo}
    \vspace*{\fill}

    \hspace{.45\textwidth}
    \begin{minipage}{.5\textwidth}
      \SingleSpacing
      \imprimirpreambulo
    \end{minipage}%
    \vspace*{\fill}
  \end{center}

  % [A FAZER: Ajuste a data de aprovação. \today usa a data da compilação.]
  Trabalho aprovado. \imprimirlocal, \today:
  \vspace*{2cm}

  % [A FAZER: Preencha com os nomes dos membros da sua banca examinadora.]
  \assinatura{\textbf{\imprimirorientador} \\ Orientador(a)}
  \assinatura{\textbf{Nome Completo Membro da Banca 1} \\ Instituição}
  \assinatura{\textbf{Nome Completo Membro da Banca 2} \\ Instituição}
  % \assinatura{\textbf{Nome Completo do Membro da Banca 3} \\ Instituição} % Descomente se necessário

\end{folhadeaprovacao}


% -----------------------------------------------------------------------------------
% -- DEDICATÓRIA (Opcional)
% -----------------------------------------------------------------------------------
% [A FAZER: Escreva sua dedicatória aqui ou comente/remova todo o ambiente.]
\begin{dedicatoria}
	\vspace*{\fill} % Centraliza o texto verticalmente
	\flushright    % Alinha o texto à direita
	\textit{DEDICATÓRIA: Agradeço aos ..., pelo apoio e incentivo incondicional...}
\end{dedicatoria}


% -----------------------------------------------------------------------------------
% -- AGRADECIMENTOS (Opcional)
% -----------------------------------------------------------------------------------
% [A FAZER: Escreva seus agradecimentos aqui ou comente/remova todo o ambiente.]
\begin{agradecimentos}
Agradeço a todos que contribuíram para a realização deste trabalho. Em especial, ao meu orientador, Prof. Dr. [Nome do Orientador], pelo suporte e orientação ao longo de todo o processo.

\end{agradecimentos}


% -----------------------------------------------------------------------------------
% -- EPÍGRAFE (Opcional)
% -----------------------------------------------------------------------------------
% [A FAZER: Adicione uma epígrafe aqui ou comente/remova todo o ambiente.]
\begin{epigrafe}
	\vspace*{\fill}
	\begin{flushright}
		\textit{``A educação é a arma mais poderosa que você pode usar para mudar o mundo."} \\
		\textendash{} Nelson Mandela
	\end{flushright}
\end{epigrafe}


% -----------------------------------------------------------------------------------
% -- RESUMO EM PORTUGUÊS (Obrigatório)
% -----------------------------------------------------------------------------------
\begin{resumo}
	% [A FAZER: Escreva o resumo do seu trabalho em português aqui.]
	O resumo deve apresentar de forma concisa o tema do trabalho, o problema de pesquisa, os objetivos, a metodologia utilizada e os principais resultados ou conclusões alcançadas. Ele é uma parte crucial do trabalho, pois fornece uma visão geral que permite ao leitor compreender rapidamente o escopo e a contribuição da sua pesquisa. O resumo deve ser escrito em um único parágrafo, contendo entre 150 e 500 palavras, conforme a ABNT.

	\vspace{\onelineskip} % Adiciona uma linha em branco antes das palavras-chave
	\noindent % Impede a indentação da linha das palavras-chave
	\textbf{Palavras-chave}: Engenharia de Software; Metodologia Ágil; Qualidade de Software.
\end{resumo}


% -----------------------------------------------------------------------------------
% -- RESUMO EM LÍNGUA ESTRANGEIRA (Obrigatório)
% -----------------------------------------------------------------------------------
\begin{resumo}[Abstract] % O título do resumo no idioma desejado
	\begin{otherlanguage*}{english} % Define o idioma para este ambiente

		% [A FAZER: Escreva o resumo do seu trabalho em inglês aqui.]
		The abstract should concisely present the theme of the work, the research problem, objectives, the methodology used, and the main results or conclusions reached. It is a crucial part of the work, as it provides an overview that allows the reader to quickly understand the scope and contribution of your research. The abstract should be written in a single paragraph, containing between 150 and 500 words, according to ABNT standards.

		\vspace{\onelineskip}
		\noindent
		\textbf{Keywords}: Software Engineering; Agile Methodology; Software Quality.

	\end{otherlanguage*}
\end{resumo}